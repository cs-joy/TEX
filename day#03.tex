%% Brackets, Tables and Arrays

\documentclass[11pt]{article}
\usepackage{amssymb, amsmath, amsfonts}
\usepackage{float}
\parindent 0px
\pagestyle{empty}

\begin{document}

$$::Brackets::$$\\[20pt]
The distributive property states that $a(b+c)=ab+ac$, for all $a, b, c \in \mathbb{R}$.\\[6pt]
The equivalence class of $a$ is $[a]$.\\[6pt]
The set $A$ is defined to be $\{1,2,3\}$.\\[6pt]
The movie ticket is $\$11.50$.

$$2\left(\frac{1}{x^2-1}\right)$$
$$2\left[\frac{1}{x^2-1}\right]$$
$$2\left\{\frac{1}{x^2-1}\right\}$$

Angular bracket
$$2\left \langle \frac{1}{x^2-1}\right \rangle $$

Absolute value symbol
$$2\left | \frac{1}{x^2-1}\right | $$

$$\left.\frac{dy}{dx}\right|_{x=1}$$

$$\left(\frac{1}{1+\left(\frac{1}{1+x}\right)}\right)$$




Tables:\\

\begin{tabular}{|c||c|c|c|c|c|}
% c means element position in center of the single box and the nuber of c is equal to the number of column of the table.
%% \begin{tabular}{llllll}
% l means element position in left of the single box and the nuber of l is equal to the number of column of the table.
%% \begin{tabular}{rrrrrr}
% r means element position in right of the single box and the nuber of r is equal to the number of column of the table.

%x 1 2 3 4 5
\hline
$x$ & 1 & 2 & 3 & 4 & 5 \\ \hline
$f(x)$ & 10 & 11 & 12 & 13 & 14 \\ \hline
\end{tabular}


\vspace{1cm}


\begin{table}[H]
\centering
\def \arraystretch{1.5}
\begin{tabular}{|c||c|c|c|c|c|}
\hline
$x$ & 1 & 2 & 3 & 4 & 5 \\ \hline
$f(x)$ & $\frac{1}{2}$ & 11 & 12 & 13 & 14 \\ \hline
\end{tabular}
\caption{These values represent the function of $f(x)$.}
\end{table}





\begin{table}[H]
\centering
\def \arraystretch{1.5}
\caption{The relationship between $f$ and $f'$.}
\begin{tabular}{|l|p{4in}|}
\hline
$f(x)$ & $f'(x)$ \\ \hline
$x>0$ & The function $f(x)$ if increasing.The function $f(x)$ if increasing.The function $f(x)$ if increasing.The function $f(x)$ if increasing.The function $f(x)$ if increasing. \\ \hline
\end{tabular}
\end{table}



Arrays::
% \, this is need when we want to use a space between two things
\begin{align}
5x^2\,\text{place your words here} \\
5x^2\text{ your words}\\
5x^2-9=x+3\\
5x^2-x-12=0
\end{align}

\begin{align*}
5x^2-9&=x+3\\
5x^2-x-12&=0\\
&=12-x-5x^2
\end{align*}

\begin{align}
5x^2-9&=x+3\\
5x^2-x-12&=0\\
&=12-x-5x^2
\end{align}























\end{document}
