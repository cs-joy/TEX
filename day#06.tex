\documentclass[11pt, letterpaper]{article} % letterpaper or a4paper
%\usepackage{fullpage} % fullpage or geometry
%\usepackage{geometry}
%\usepackage[margin=1in]{geometry}
%\usepackage[top=1in, bottom=1in, left=0.5in, right=0.5in]{geometry}
%\usepackage[top=1in, bottom=1in, left=0.5in, right=0.5in, paperwidth=8in, paperheight=7in]{geometry}
\usepackage[margin=1in]{geometry}
\usepackage{amsfonts, amssymb, amsmath}

\usepackage{tikz, pgfplots}

% define
\def\eq1{y=\dfrac{x}{3x^2+x+1}}

% create new command
\newcommand{\set}[1]{\setlength\itemsep{#1em}}

\newcommand\calculator{\tikz{\node (c)[inner sep=0pt, draw, fill=black, anchor=south west]{\phantom{N}};
\begin{scope}[x=(c.south east), y=(c.north west)] \fill[white](.1,.7) rectangle(.9,.9);\foreach \x in {.1, .33, .55, .79}{\foreach \y in {.1, .24, .38, .53}{\fill[white] (\x,\y) rectangle +(.11, .07);}}
\end{scope}}}
\def\calcicon#1{\noindent#1 \calculator\}

\begin{document}

\textbf{Critical Thinking Questions}
\begin{enumerate}
\set{1.2}
\item Let's examine the function: $y=\frac{x}{3x^2+x+1}$.
\item \calculator Let's examine the function: $\eq1$.
\item This is symbol for the set of all real numbers: $\mathbb{R}$.
\item This is symbol for the set of integers: $\mathbb{Z}$.
\item This is symbol for the set of rationals: $\mathbb{Q}$.
\item Is it possible for a sequence to converge to two different numbers? If so, give an example. If not, explain why not?
\item Explain how to use partial sums to determine if a series converges or diverges. Give an example.
\item Explain why $\int\limits_{1}^{\infty} f(x)\,dx$ and $\sum\limits_{n=1}^{\infty} a_n$ need not converge to the same value, even if they are both convergent.
\item In your words, explain the Alternating Series Remainder Theorem. How is this theorem useful?
\item Explain the difference between absolute and conditional convergence. Give an example of each.
\item The Ratio test is inclusive if 
\end{enumerate}

\end{document}